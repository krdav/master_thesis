% $Log: abstract.tex,v $
% Revision 1.1  93/05/14  14:56:25  starflt
% Initial revision
% 
% Revision 1.1  90/05/04  10:41:01  lwvanels
% Initial revision
% 
%
%% The text of your abstract and nothing else (other than comments) goes here.
%% It will be single-spaced and the rest of the text that is supposed to go on
%% the abstract page will be generated by the abstractpage environment.  This
%% file should be \input (not \include 'd) from cover.tex.
Adaptive immunity is a highly active branch of biology dealing with issues that are relevant for understanding and treating a wide range of human diseases as well as basic understanding of animal physiology.
B cell and their B cell receptors (BCRs) take a central role in the adaptive immune system e.g. in vaccine immunity caused by antibodies, the secreted form of BCRs.
The adaptivity of BCRs stem from the darwinian evolution which they undergo to specifically neutralize foreign antigens, but only recently high throughput sequencing (HTS) has enabled to study this evolutionary process.
An important objective of these HTS studies is to reconstruct the phylogeny of the evolutionary process, such that it is possible to understand the developmental path of immunity.
Tools and theory are already well tested in the field of phylogeny, however most have been developed to study population genetics or evolution of organisms over millions of years, while BCR evolution occurs under very different conditions.
We simulate BCR evolution under both a neutral model and a model we derive to represent realistic BCR sequence evolution, in which the fitness function is coupled to antigen affinity.
We simulate data to recapitulate summary statistics of real BCR data, use a number of different phylogenetic methods to infer the simulated phylogenies and finally perform a validation using topological similarity and a novel metric we define to capture the correctness of an ancestral sequence reconstruction.
Our results show that phylogenetic inference is robust to both simple and advanced simulations, and when sampling sequences at a single time point, sequence reconstruction is largely insensitive to the different methods tested.
This indicates that inferring BCR phylogeny might not be as hard, regardless of its unconventional nature.
