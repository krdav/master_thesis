Adaptive immunology is a fundamental aspect of human physiology, and essential to understanding and treating a wide range of human diseases.
B cells and their B cell receptors (BCRs) have a central role in the adaptive immune system e.g.\ in vaccine-induced immunity caused by antibodies and autoimmune diseases.
The adaptability of BCRs stems from a Darwinian evolutionary process in which B cells are matured in the germinal center (GC) to specifically neutralize foreign antigen; it recently become possible to study this in detail using high throughput sequencing (HTS).
An important objective of these HTS studies is to reconstruct the phylogeny of the GC evolutionary process such that it is possible to understand the developmental path of immunity.
General-purpose tools and theory for inferring phylogenies are already developed and tested, however most have been developed to study population genetics or evolution of organisms over millions of years, while BCR evolution occurs under very different conditions.
We have setup a simulation framework of BCR evolution under both a neutral and an affinity selection model designed to represent realistic BCR sequence evolution where the fitness function is coupled to antigen affinity.
We simulate data to recapitulate summary statistics of real BCR data and use a number of different phylogenetic methods to infer the simulated phylogenies.
Using true and inferred phylogenies we perform a validation to assess the topological similarities and the correctness of the ancestral sequence reconstruction, measured by a novel metric we call COAR.
Our results show that phylogenetic inference is robust to both simple and advanced simulations and that sequence reconstruction is surprisingly insensitive to the choice of methods in our tests.
This confirms the validity of using standard phylogenetic inference on GC evolved BCR sequences, regardless of its unconventional evolutionary nature.
We close by discussing further opportunities to improve methods.
